\section*{Entrega}
\begin{itemize}
\item \textbf{Tarea y} \texttt{README.md}
  \begin{itemize}
      \item \textbf{Fecha y hora oficial (sin atraso):} \textbf{ACTUALIZAR}
      \item \textbf{Fecha y hora máxima (2 días de atraso):} \textbf{ACTUALIZAR}
      \item \textbf{Lugar:} Repositorio personal de GitHub --- Carpeta: \texttt{\folder}. \\ El código debe estar en la rama (\textit{branch}) por defecto del repositorio: \texttt{main}.
      
      \item \textbf{Pauta de corrección}: \link{}{\textbf{ACTUALIZAR}}.
      
      \item \textbf{Bases generales de tareas (descuentos)}: \link{https://github.com/IIC2233/Syllabus/blob/main/Tareas/Bases\%20Generales\%20de\%20Tareas\%20-\%20IIC2233.pdf}{en este enlace}.
  \end{itemize}

  \item \textbf{Ejecución de tarea}: La tarea será ejecutada \textbf{únicamente} desde la terminal del computador. Además, durante el proceso de corrección, se cambiará el nombre de la carpeta ``\texttt{\homeworkname{}}'' por otro nombre y se ubicará la terminal dentro de dicha carpeta antes de ejecutar la tarea. \textbf{Los \textit{paths} relativos utilizados en la tarea deben ser coherentes con esta instrucción.}

\end{itemize}

\section*{Objetivos}
\begin{itemize}
    \item Aplicar conceptos de programación orientada a objetos (POO) para modelar y resolver un problema complejo.
    
    \item Utilizar \textit{properties}, clases abstractas y polimorfismo como herramientas de modelación. \textbf{ACTUALIZAR}
    
    \item Procesar \textit{input} del usuario de forma robusta, manejando potenciales errores de formato.
    
    \item Trabajar con archivos de texto para leer y procesar datos.
    
    \item Familiarizarse con el proceso de entrega de tareas y uso de buenas prácticas de programación.
\end{itemize}

\clearpage
{
  \hypersetup{linkcolor=black}
  \small
  \tableofcontents
}
\clearpage