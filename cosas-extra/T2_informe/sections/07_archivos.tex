\label{sec:archivos}
Para el desarrollo de \programa\ será necesaria la lectura y carga de archivos. Estos contendrán información sumamente relevante para la ejecución correcta de tu programa.

\subsection{Árboles por dificultad}
Archivos de texto \texttt{\mil{arboles_faciles.txt}}, \texttt{\mil{arboles_medios.txt}}, \texttt{\mil{arboles_dificiles.txt}} 
ubicados en la carpeta \texttt{data/}, que definen la estructura de los arboles que el jugador y el enemigo podrán seleccionar según la dificultad seleccionada.

\begin{figure}[H]
    \centering
    \begin{minipage}{0.8\textwidth}
        \begin{python}
{
    Roble Guerrer;
    Ficus;
    {
        Hyedrid;
        Paalm;
        Alovelis
    };
    {
        Pine;
        Cactoos;
        Cactoos;
        {
            Alovelis;
            Hyedrid;
            Hyedrid
        };
        Celery
    };
    Celery
};
{
    Sauce Veloz;
    Ficus;
    {
        Hyedrid;
        Paalm;
        Alovelis
    }
}
        \end{python}
        \caption{Ejemplo de archivo \mil{arboles_faciles.txt}}
        \label{fig:facil}
    \end{minipage}
\end{figure}


Un cuadro que explica mejor la estructura de los árboles, con sus respectivas ramas, es el siguiente:


\begin{figure}[H]
    \centering
    \begin{minipage}{0.8\textwidth}
        \begin{verbatim}
    Roble Guerrer
    +-- Ficus
    +-- Hyedrid
    |   +-- Paalm
    |   \-- Alovelis
    +-- Pine
    |   +-- Cactoos
    |   +-- Cactoos
    |   +-- Alovelis
    |   |   +-- Hyedrid
    |   |   \-- Hyedrid
    |   \-- Celery
    \-- Celery
    
    Sauce Veloz
    +-- Ficus
    \-- Hyedrid
        +-- Paalm
        \-- Alovelis
        \end{verbatim}
        \caption{Estructura de los árboles del archivo \mil{arboles_faciles.txt}}
        \label{fig:ramas}
    \end{minipage}
\end{figure}

Cabe destacar, cada árbol tiene un nombre único que lo identifica el cual aparece en el inicio del árbol. 
Además, la estructura de los árboles contiene el nombre de cada una de sus ramas hijas, 
donde será tarea de ustedes realizar la carga de las respectivas ramas hijas
con sus atributos. En la siguiente sección se detallan los atributos de cada rama.

\subsection{Ramas Disponibles}
\label{subsec:ramas}

Archivo \texttt{ramas.txt} que define los tipos de ramas, con sus atributos y efectos ya mencionados en la sección \ref{subsubsec:Tipos de Ramas}.

\begin{table}[H]
\centering
\label{tab:ramas}
\begin{tabular}{|l|l|}
\hline
\textbf{Atributo} & \textbf{Tipo} \\ \hline
\mil{nombre_rama} & String \\ \hline
\mil{puntos_de_vida} & Integer \\ \hline
\mil{resistencia} & Float \\ \hline
\mil{daño_base} & Integer \\ \hline
\mil{item} & Item|None \\ \hline
\mil{efecto} & Callable aplicado al atacar/defender \\ \hline
\end{tabular}
\caption{Especificación de ramas}
\end{table}

\begin{figure}[H]
    \centering
    \begin{minipage}{0.8\textwidth}
        \begin{python}
Ficus;1600;0.85;45
Celery;1200;0.75;35
Hyedrid;1000;0.65;75
Paalm;2000;0.90;40
Alovelis;1800;0.80;25
Pine;900;0.60;60
Cactoos;800;0.50;55
        \end{python}
        \caption{Ejemplo de archivo \mil{ramas.txt}}
        \label{fig:ramas}
    \end{minipage}
\end{figure}

Es importante mencionar que el atributo \mil{item} puede ser \mil{None} o un ítem de la tienda. En caso de que el ítem sea \mil{None}, no se aplicará ningún efecto al atacar o defender. En caso contrario, se aplicará el efecto del ítem al atacar o defender.
En la siguiente sección se detallan los ítems disponibles en la tienda.

\subsection{Ítems de la Tienda}
\label{subsec:items}

Archivo \texttt{items.txt} que define los ítems disponibles en la tienda, donde cada uno aplicará un \texttt{buff} o \texttt{debuff} a la rama que lo posea. 

\begin{table}[H]
\centering
\label{tab:items}
\begin{tabular}{|l|l|}
\hline
\textbf{Atributo} & \textbf{Tipo} \\ \hline
\mil{nombre_item} & String \\ \hline
\mil{ataque} & Fórmula aplicada (e.g. \mil{"ataque + 2"}) \\ \hline
\mil{defensa} & Fórmula aplicada (e.g. \mil{"defensa + 2"}) \\ \hline
\mil{vida_maxima} & Fórmula aplicada (e.g. \mil{"vida_maxima + 2"})\\ \hline
\mil{precio} & Integer \\ \hline
\end{tabular}
\caption{Especificación de ítems}
\end{table}

\begin{figure}[H]
    \centering
    \begin{minipage}{0.8\textwidth}
        \begin{python}
Flor;0;0;50
Flor Dorada;0;0.05;60
Panal de Abejas;30;0;10
Nido de Pajaros;100;-0.1;0
        \end{python}
        \caption{Ejemplo de archivo \mil{items.txt}}
        \label{fig:items}
    \end{minipage}
\end{figure}

\subsection{Ítems malignos}

Archivo \texttt{\mil{items_malignos.txt}} que define los ítem malignos (\texttt{plagas}) que pueden llegar a afectar
a cualquiera de las ramas del árbol del jugador o del enemigo. Cuando el jugador \textbf{Transcurre el tiempo}, existe la posibilidad de que alguna de sus ramas
o las del enemigo (según la \mil{resistencia_a_plagas} de cada rama), se vea afectada por un ítem maligno. En este caso, el ítem maligno se aplicará a la rama afectada,
ocupando el \texttt{Slot} de ítem de la respectiva rama. En caso de que la rama ya tenga un ítem, el ítem maligno sobreescribirá el ítem existente.

\begin{table}[H]
\centering
\label{tab:items}
\begin{tabular}{|l|l|}
\hline
\textbf{Atributo} & \textbf{Tipo} \\ \hline
\mil{nombre_item} & String \\ \hline
\mil{ataque} & Fórmula aplicada (e.g. \mil{"ataque + 2"}) \\ \hline
\mil{defensa} & Fórmula aplicada (e.g. \mil{"defensa + 2"}) \\ \hline
\mil{vida_maxima} & Fórmula aplicada (e.g. \mil{"vida_maxima + 2"})\\ \hline
\end{tabular}
\caption{Especificación de ítems}
\end{table}

\begin{figure}[H]
    \centering
    \begin{minipage}{0.8\textwidth}
        \begin{python}
Termitas;0;-0.3;0
Zancudos;0;0;-100
        \end{python}
        \caption{Ejemplo de archivo \mil{items_malignos.txt}}
        \label{fig:items}
    \end{minipage}
\end{figure}

Cabe destacar que los ítems malignos no tienen precio, ya que no se pueden comprar en la tienda.

\subsection{parametros.py}
Para esta tarea, se requiere la creación de un archivo \texttt{parametros.py} donde deberás \textbf{completar todos los parámetros mencionados a lo largo del enunciado}. Dichos parámetros se presentarán en \const{ESTE\_FORMATO} y en ese color. Además, es fundamental incluir cualquier valor constante necesario en tu tarea, así como cualquier tipo de \textit{path} utilizado.

Un parámetro siempre tiene que estar nombrado de acuerdo a la función que cumplen, por lo tanto, asegúrate de que sean descriptivos y reconocibles. Ejemplos de parametrización:

\begin{python}
CINCO = 5  # mal parámetro
DINERO_MAXIMO = 3000  # buen parámetro
PROBABILIDAD_TORMENTA = 0.2  # buen parámetro
\end{python}

Si necesitas agregar algún parámetro que varíe de acuerdo a otros parámetros, una correcta parametrización sería la siguiente:

\begin{python}
PI = 3.14
RADIO_CIRCUNFERENCIA = 3
AREA_CIRCUNFERENCIA = PI * (RADIO_CIRCUNFERENCIA ** 2)
\end{python}

Dentro del archivo \textbf{parametros.py}, es obligatorio que hagas uso de todos los parámetros almacenados y los importes correctamente. Cualquier información no relacionada con parámetros almacenada en este archivo resultará en una penalización en tu nota. Recuerda que no se permite el \link{https://es.wikipedia.org/wiki/Codificaci\%C3\%B3n_r\%C3\%ADgida}{hard-coding}\footnote{\textit{Hard-coding} es la práctica de ingresar valores directamente en el código fuente del programa en lugar de parametrizar desde fuentes externas.}, ya que esta práctica se considera incorrecta y su uso conllevará una reducción en tu calificación.