\label{sec:importante}

% Para esta tarea, el carácter funcional del programa será el pilar de la corrección, es decir, \textbf{sólo se corrigen tareas que se puedan ejecutar}. Por lo tanto, se recomienda hacer periódicamente pruebas de ejecución de su tarea y \textit{push} en sus repositorios, corroborando que cada archivo de \textit{tests} que les pasamos corra en un tiempo acotado de 5 minutos, en caso contrario se asumirá un resultado incorrecto.

En el \link{}{ACTUALIZAR} se encuentra la distribución de puntajes. En esta señalará con color \textbf{amarillo} cada ítem que será evaluado a nivel funcional y de código, es decir, aparte de que funcione, se revisará que el código esté bien confeccionado. Todo aquel que no esté pintado de amarillo será evaluado si y sólo si se puede probar con la ejecución de su tarea.

\textbf{Importante}: Todo ítem corregido por el cuerpo docente será evaluado únicamente de forma ternaria: cumple totalmente el ítem, cumple parcialmente o no cumple con lo mínimo esperado. Finalmente, todos los descuentos serán asignados manualmente por el cuerpo docente respetando lo expuesto en \link{https://github.com/IIC2233/Syllabus/blob/main/Tareas/Bases\%20Generales\%20de\%20Tareas\%20-\%20IIC2233.pdf}{el documento de bases generales}.

Para terminar, si durante la realización de tu tarea se te presenta algún problema o situación que pueda afectar tu rendimiento, no dudes en contactar al ayudante de Bienestar de tu sección. El correo está en el  \link{https://github.com/IIC2233/Syllabus/wiki/8.-\%F0\%9F\%92\%8C-Contacto-del-curso}{siguiente enlace}.
