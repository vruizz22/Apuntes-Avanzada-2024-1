% \clearpage
\section{Restricciones y alcances}
\begin{itemize} 
    \item Esta tarea es \textbf{estrictamente individual}, y está regida por el \link{https://www.ing.puc.cl/wp-content/uploads/2015/08/cdigo-de-honor.pdf}{Código de honor de Ingeniería}.
    
    \item Tu programa debe ser desarrollado en Python 3.11.X  con X mayor o igual a 7.
    
    \item Tu programa debe estar compuesto por uno o más archivos de extensión \texttt{.py} que estén correctamente ordenados por carpeta. \textbf{No se revisará archivos en otra extensión como\texttt{.ipynb}}.

    \item Toda el código entregado debe estar contenido en la carpeta y rama (\textit{branch}) indicadas al inicio del enunciado. Ante cualquier problema relacionado a esto, es decir, una carpte distinta a \mil{T2} o una rama distinta a \texttt{main}, se recomienda preguntar en las \link{https://github.com/IIC2233/syllabus/issues}{\textit{issues} del foro}.
    % \hernan{Agregué este texto, pero mejor si es iterado por más manos.}

    \item Si no se encuentra especificado en el enunciado, supón que el uso de cualquier librería Python está prohibido. Pregunta en la \textit{issue} especial del \link{https://github.com/IIC2233/syllabus/issues}{foro} si es que es posible utilizar alguna librería en particular.
    \item Debes adjuntar un \textbf{único archivo markdown}, llamado \texttt{README.md}, \textbf{conciso y claro}, donde describas los alcances de tu programa, cómo correrlo, las librerías usadas, los supuestos hechos, y las referencias a código externo. El no incluir este archivo, incluir un readme vacío o el subir más de un archivo \texttt{.md}, conllevará un \link{https://github.com/IIC2233/Syllabus/blob/main/Tareas/Bases\%20Generales\%20de\%20Tareas\%20-\%20IIC2233.pdf}{descuento} en tu nota.
    % \gery{cambiar}\hernan{ya está cambiado.}

    \item Esta tarea se debe desarrollar \textbf{exclusivamente} con los contenidos liberados al momento de publicar el enunciado. No se permitirá utilizar contenidos que se vean posterior a la publicación de esta evaluación. 

    \item Se encuentra estrictamente prohibido citar código que haya sido publicado \textbf{después de la liberación del enunciado}. En otras palabras, solo se permite citar contenido que ya exista previo a la publicación del enunciado. Además, se encuentra estrictamente prohibido el uso de herramientas generadoras de código para el apoyo de la evaluación.
    
    \item Cualquier aspecto no especificado queda a tu criterio, siempre que no pase por sobre otro que sí sea especificado por enunciado.
\end{itemize}
\medskip
\textbf{Las tareas que no cumplan con las restricciones del enunciado obtendrán la calificación mínima ($1.0$).}